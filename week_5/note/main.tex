% modified from NeurIPS 2022 Latex template
% authors: Yixin Zhu (yixin.zhu@pku.edu.cn), Liangru Xiang
% v1.0: 2022.07.21

\documentclass{article}
\PassOptionsToPackage{numbers,compress}{natbib}
\usepackage[final]{template22}
\input{config.tex}

\title{The learning of week 4}

\author{%
  SuYingrui \\
  Department of Madicine\\
  Peking University\\
  \texttt{2310306221@stu.pku.edu.cn} \\
  % examples of more authors
  % \And
  % Coauthor \\
  % Affiliation \\
  % Address \\
  % \texttt{email} \\
  % \AND
  % Coauthor \\
  % Affiliation \\
  % Address \\
  % \texttt{email} \\
  % \And
  % Coauthor \\
  % Affiliation \\
  % Address \\
  % \texttt{email} \\
  % \And
  % Coauthor \\
  % Affiliation \\
  % Address \\
  % \texttt{email} \\
}

\begin{document}
\maketitle

\section{The list is additive}

Such as the list [1,2,3] + [5]

It returns the [1,2,3,5]

And the [1,2,3]*2  returns the [1,2,3,1,2,3]

\section{What is the range(3,6) returns}

It returns the range(3, 6), yes it is itself

So we should wake up to it that the range is a class, and it is iterable and memory efficient, and it wouldn't returns all the numbers

\section{The function of round()}

The function can make the number of int type

\section{The 11 ppt's function large(s,n)}

The function is so delicate. The most important thing in designing recursion function is the cases should be fully considered.

\section{The use of list}

a = [4,5,6]

a[1][0]

Then it returns the error:'int' object is not subscriptable






\end{document}